\documentclass{article}
\usepackage{graphicx} % Required for inserting images
\usepackage{makecell}

\title{CSE 415 Assignment 2 Report: \\
Evaluating Search Algorithms and Heuristics}
\author{Your Names Here }
\date{January 2025}

\begin{document}

\maketitle

\section{Introduction}
We are Aidan Lee and Dane Grassy. 
This is our report for Assignment 2 covering both blind search algorithms
and heuristic search.

\section{Report on Part A: Problem Formulation and Blind Search Algorithms}

\subsection{Part A Step 4 (c)}
In this method, there are five lines of code containing "return False". Explain what is going on in the 4th one of these lines. Put your answer into your Part C Report file ("SearchReport.pdf") under the heading "Part A Step 4 (c)."

Humans must not be outnumbered on either side:
\\
\\ if h-remaining $>$ 0 and h-remaining $<$ r-remaining: return False
\\ This code tests whether or not an operation would create a valid state or not. After calculating how many robots and humans would be left after the operation, this line tests if it is a valid state, where humans cannot be outnumbered. This line would return false in any invalid state where humans are outnumbered and at least one human is on the side, but would not return false if there were no humans. 



\subsection{Part A Step 8}

(your answer for the question in Part A step 8 goes into the table below, as
well as the path details in 2.3 and explanations in 2.4.)

The paths are not required in the report for the entries marked "skip."

{\flushleft
\begin{tabular}{|l|p{2cm}|p{2cm}|p{3cm}|}
\hline
\parbox{3.5cm}{Problem and\\ Algorithm} & Path Found & Path length & \#Nodes Expanded \\
\hline
\makecell[l]{Humans, Robots\\ and Ferry / DFS} & (skip) & 9 & 10 \\
\hline
\makecell[l]{Humans, Robots\\ and Ferry / BreadthFS} & & 7 & 10 \\
\hline
\makecell[l]{Farmer, Fox, Chicken\\ and Grain/ DFS} & & 7 & 7\\
\hline
\makecell[l]{Farmer, Fox, Chicken\\ and Grain/ BreadthFS} & & 7&9 \\
\hline
\makecell[l]{4-Disk Towers of\\ Hanoi/DFS} & (skip) & 40 & 40\\
\hline
\makecell[l]{4-Disk Towers of\\ Hanoi/BreadthFS} & &15 & 70\\
\hline
\end{tabular}}

\subsection{Part A Step 8, Path details}
 Paths found (if not shown in the table).  Copy the state sequences
 obtained from the search algorithm on the requested problems.

 \begin{itemize}
 \item HRF/BreadthFS: Congratulations on successfully guiding the humans and robots across the creek!
Solution path: 

 H on left:3
 R on left:3
   H on right:0
   R on right:0
 ferry is on the left.


 H on left:2
 R on left:2
   H on right:1
   R on right:1
 ferry is on the right.


 H on left:3
 R on left:2
   H on right:0
   R on right:1
 ferry is on the left.


 H on left:0
 R on left:2
   H on right:3
   R on right:1
 ferry is on the right.


 H on left:2
 R on left:2
   H on right:1
   R on right:1
 ferry is on the left.


 H on left:0
 R on left:1
   H on right:3
   R on right:2
 ferry is on the right.


 H on left:1
 R on left:1
   H on right:2
   R on right:2
 ferry is on the left.


 H on left:0
 R on left:0
   H on right:3
   R on right:3
 ferry is on the right.
 \item FFCG/DFS: Solution path: 

 Farmer on the left
 Fox on the left
 Chicken on the left
 Grain on the left

 Farmer on the right
 Fox on the left
 Chicken on the right
 Grain on the left

 Farmer on the left
 Fox on the left
 Chicken on the right
 Grain on the left

 Farmer on the right
 Fox on the right
 Chicken on the right
 Grain on the left

 Farmer on the left
 Fox on the right
 Chicken on the left
 Grain on the left

 Farmer on the right
 Fox on the right
 Chicken on the left
 Grain on the right

 Farmer on the left
 Fox on the right
 Chicken on the left
 Grain on the right

 Farmer on the right
 Fox on the right
 Chicken on the right
 Grain on the right
 \item FFCG/BreadthFS: 
 Solution path: 

 Farmer on the left
 Fox on the left
 Chicken on the left
 Grain on the left

 Farmer on the right
 Fox on the left
 Chicken on the right
 Grain on the left

 Farmer on the left
 Fox on the left
 Chicken on the right
 Grain on the left

 Farmer on the right
 Fox on the right
 Chicken on the right
 Grain on the left

 Farmer on the left
 Fox on the right
 Chicken on the left
 Grain on the left

 Farmer on the right
 Fox on the right
 Chicken on the left
 Grain on the right

 Farmer on the left
 Fox on the right
 Chicken on the left
 Grain on the right

 Farmer on the right
 Fox on the right
 Chicken on the right
 Grain on the right
 \item 4-Disk TOH/BreadthFS:
 Solution path: 
\\ $\ [[4, 3, 2, 1] ,[] ,[]]$
\\ $\ [[4, 3, 2] ,[1] ,[]]$
\\ $\ [[4, 3] ,[1] ,[2]]$
\\ $\ [[4, 3] ,[] ,[2, 1]]$
\\ $\ [[4] ,[3] ,[2, 1]]$
\\ $\ [[4, 1] ,[3] ,[2]]$
\\ $\ [[4, 1] ,[3, 2] ,[]]$
\\ $\ [[4] ,[3, 2, 1] ,[]]$
\\ $\ [[] ,[3, 2, 1] ,[4]]$
\\ $\ [[] ,[3, 2] ,[4, 1]]$
\\ $\ [[2] ,[3] ,[4, 1]]$
\\ $\ [[2, 1] ,[3] ,[4]]$
\\ $\ [[2, 1] ,[] ,[4, 3]]$
\\ $\ [[2] ,[1] ,[4, 3]]$
\\ $\ [[] ,[1] ,[4, 3, 2]]$
\\ $\ [[] ,[] ,[4, 3, 2, 1]]$
 \end{itemize}

 \subsection{Part A Step 8,  Explanations of Certain Differences, Using Towers-of-Hanoi  }

\begin{paragraph}
(i. Why the maximum length of the OPEN list is more for one algorithm
than the other)
\\ The maximum length of the OPEN list for DFS is 7, and the maximum length of the OPEN list for BFS is 16. BFS has a larger OPEN list, as it adds all possible states at each depth to the open list first, while DFS will explore each path linearly, and remove after checking the entire depth. This results in BFS having a larger open list. 



\end{paragraph}
\begin{paragraph}
(ii. Why the solution PATH length is different for one algorithm from that of the other. )
\\ BFS has a dramatically smaller path length. This is because for ItrBFS, with 4 objects that can each have 3 states, it is able to check the paths of adding discs to multiple pegs, while DFS takes longer to add discs to other pegs, and will create a longer solution path. 

\end{paragraph}

% -----------------------------
\newpage
\section{Report on Part B: Heuristics for the Eight Puzzle}

(Your results for Part B should be reported in the table below.)


\subsection{Results with Heuristics for the Eight Puzzle}

{\flushleft
\begin{tabular}{|c|l|c|l|c|c|c|}
\hline
Puzzle & Heuristic & Solved? & \# Soln Edges & Soln Cost & \# Expanded & Max Open\\
\hline
A & none (UCS) & Y & & & & \\
\hline
A & Hamming & Y & & & & \\
\hline
A & Manhattan & Y & & & & \\
\hline
B & none (UCS) & & & & & \\
\hline
B & Hamming &  & & & & \\
\hline
B & Manhattan & Y & & & & \\
\hline
C & none (UCS) & & & & & \\
\hline
C & Hamming &  & & & & \\
\hline
C & Manhattan & Y & & & & \\
\hline
D & none (UCS) & & & & & \\
\hline
D & Hamming &  & & & & \\
\hline
D & Manhattan & Y & & & & \\
\hline

\end{tabular} }

\begin{verbatim}
Puzzle A: [3,0,1,6,4,2,7,8,5]
Puzzle B: [3,1,2,6,8,7,5,4,0]
Puzzle C: [4,5,0,1,2,8,3,7,6]
Puzzle D: [0,8,2,1,7,4,3,6,5]
\end{verbatim}

\subsection{(Optional) Evaluating Our Custom Heuristics}

Describe your custom heuristic here.  What is the underlying intuition for it?
Is it admissible? Why or why not, or why is it difficult to determine if that
is the case.  How would you compare its computational cost with that of
the Hamming heuristic and the Manhattan distance heuristic?

What puzzles did you try it on, and how did it compare?
You may add rows to your table above to support your answer about comparison.
(Give your heuristic an appropriate short name to identify it in the table.)

\newpage
\section{Partnership Retrospective}

\subsection{Partnership?}
Did you work in a partnership? (yes or no).

If so, who were the partners (repeating your names from below the title on the first page)?

\subsection{Collaboration}
Also if so, how did you did you divide the work of this assignment?

\subsection{Newness of the Collaboration}
If this was a new sort of experience for either of you, please mention that,
and in what way(s) it felt new.




\end{document}
